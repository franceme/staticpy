\usepackage[export]{adjustbox}
\usepackage{listings}
\usepackage{suffix}
\usepackage{calc} %https://tex.stackexchange.com/questions/30081/how-can-i-sum-two-values-and-store-the-result-in-other-variable
\usepackage{float}
\usepackage{titlesec}
\usepackage{soul}
\usepackage{amsmath,amssymb,amsfonts}
\usepackage{mathptmx}
\usepackage{graphicx}
\usepackage{textcomp}
\usepackage{fancyvrb}
\usepackage{hyperref}
\usepackage[utf8]{inputenc}
\usepackage{dirtytalk}
\usepackage{amsfonts}
\usepackage{amsmath}
\usepackage{amsthm}
\usepackage{graphicx}
%\usepackage{ulem}
\usepackage{framed}
\usepackage{microtype}
\usepackage{booktabs}
\usepackage{hyperref}
\usepackage{tabularx}
\usepackage{multirow}
\usepackage{hhline}
\usepackage[capitalise,noabbrev]{cleveref}
\usepackage{xcolor}
\usepackage[T1]{fontenc}
\usepackage{array}
\usepackage[acronym]{glossaries}
\usepackage{glossaries-extra}
\usepackage{algorithm}
\usepackage{eqparbox}
\newdimen{\algindent}
\setlength\algindent{1.5em}          % algorithmic indent=1.5em
\usepackage{xargs}      % Used for new commands with optional arguments
\usepackage{color}      % Used for custom colors in comments
\usepackage{xspace}     % Used for abbreviation spacing
\usepackage{xpunctuate} % Used for abbreviation spacing
%\usepackage[noend]{algpseudocode} %No Endif
\usepackage{algpseudocode}
\usepackage{todonotes}
\usepackage{listings}
\usepackage{pdfpages}
\def\blankpage{%
      \clearpage%
      \thispagestyle{empty}%
      \addtocounter{page}{-1}%
      \null%
      \clearpage}

\def\algoSpace{
\vspace*{.25\baselineskip}
}

\iffalse
All the JSON gets converted to Latex through the following website:
https://www.latex-tables.com/

re-wrap the tables with the following section

Use table* for two columns
\newcommand{\CMDNAME}{
\begin{table*}[!ht]
    \centering
    \caption{}
    %HERE
    \label{fig:rulescoveredgraph}
    \end{table*}
}

Single Column
\newcommand{\CMDNAME}{
\begin{table}[!ht]
    \centering
    \caption{}
    %HERE
    \label{fig:rulescoveredgraph}
    \end{table}
}
\fi
%Spacing
%http://www.emerson.emory.edu/services/latex/latex_119.html

%\algnewcommand\algorithmicforeach{\textbf{for each}}
%\algdef{S}[FOR]{ForEach}[1]{\algorithmicforeach\ #1\ \algorithmicdo}
%\algnewcommand\algorithmicendfor{\textbf{end for}}
%\algdef{S}[ENDFOR]{EndFor}{\algorithmicendfor}

%\usepackage{svg} %https://tex.stackexchange.com/questions/442077/is-it-possible-to-use-svg-images-with-overleaf

\makeatletter
\def\BState{\State\hskip-\ALG@thistlm}
\makeatother
\newcommand{\inject}[2]{#2}
\newcommand{\recall}{Recall}
\WithSuffix\newcommand\recall*{\recall \espace}
\newcommand{\precision}{Precision}
\WithSuffix\newcommand\precision*{\precision \espace}
\newcommand{\accuracy}{Accuracy}
\WithSuffix\newcommand\accuracy*{\accuracy \espace}

\newcommand{\preprintNotice}{\textcolor{red}{Preprint: Currently under review. Please do not redistribute.}}
\newcommand{\sys}{StillUndecided}
\newcommand{\ie}{{i.e.,}\xspace}
\newcommand{\eg}{{e.g.,}\xspace}
\newcommand{\etal}{{et~al\xperiod}\xspace}
\newcommand{\aka}{{a.k.a.}\xspace}
\newcommand{\etc}{{etc\xperiod}\xspace}

%% Note: Commands for custom styling of participant quotes
\newcommand{\myquote}[1]{\emph{``#1''}}                         % Italic text in quotation marks
\newcommand{\myquoteparticipant}[2]{\emph{``#1''}~(#2)}         % Italic text in quotation marks AND P# with non-breaking space
\newcommand{\mylongquote}[1]{\begin{quote}\emph{#1}\end{quote}} % Block quote in italics, no quotation marks or P#

\newcommand{\twoline}[2]{\begin{tabular}[l]{l@{}l@{}}#1\\#2\\\end{tabular}}
\newcommand{\threeline}[3]{\begin{tabular}[l]{l@{}l@{}l@{}}#1\\#2\\#3\\\end{tabular}}





%% Note: Commands for custom styling of bold, inline paragraph headings
\newcommand{\parHeading}[1]{\vspace{8px}\noindent{\textbf{#1}}}

%% Note: Custom colors for in-paper comments
\definecolor{lightpink}{RGB}{237,157,202}
\definecolor{lightred}{RGB}{210,121,121}
\definecolor{lightorange}{RGB}{230,170,50}
\definecolor{lightgold}{RGB}{210,194,121}
\definecolor{lightgreen}{RGB}{121,210,121}
\definecolor{lightaqua}{RGB}{121,206,210}
\definecolor{lightblue}{RGB}{121,124,210}
\definecolor{lightpurple}{RGB}{153,102,255}
\definecolor{red}{RGB}{178,34,34}
\definecolor{gray}{RGB}{166,166,166}

%% Note: General DO and cut commands
%\newcommand{\do}[1]{\textcolor{red}{[TODO] \emph{#1}}}
%\newcommand{\cut}[1]{\textcolor{red}{\st{#1}}}

%% Note: Labeled in-paper comments for paper authors (with and without underlined text)
\newcommandx{\guest}[3][1=]
    {\setulcolor{lightorange}{\ul{#1}} \textcolor{lightorange} %% Usage: \guest[Underline]{Name}{Comment}
    {[\textbf{#2:} #3]}}
\newcommandx{\jane}[2][1=] 
    {\setulcolor{lightgreen}{\ul{#1}} \textcolor{lightgreen}   %% Usage: \jane[(optionally) underline text]{With a comment.}
    {[\textbf{Jane:} #2]}}
    
%% Note: Section status badges to label which sections are ready (or not) for feedback 
\newcommand{\badge}[2]{\colorbox{#1}{\textcolor{white}{\textsc{#2}}}}
\newcommand{\headerBadge}[2]{
  \vspace{-15px}                  % Move badge into header
  \hspace*{\fill}                 % Align badge right
  \badge{#1}{#2}                  % Create badge
  \vspace{4px}\linebreak\noindent % Spacing for first paragraph
}
\newcommand{\complete}{\headerBadge{lightpurple}{complete}}
\newcommand{\feedbackProvided}{\headerBadge{lightgreen}{feedback provided}}
\newcommand{\readyForFeedback}{\headerBadge{lightorange}{ready for feedback}}
\newcommand{\underRevision}{\headerBadge{lightred}{actively under revision}}
\newcommand{\incomplete}{\headerBadge{red}{missing or incomplete}}

\setabbreviationstyle[acronym]{long-short}

\newcommand{\RomanNumeral}[1]{\MakeLowercase{\romannumeral #1}}
\newcommand{\RomanNumeralCaps}[1]{\MakeUppercase{\romannumeral #1}}

\newacronym{http}{HTTP}{HyperText Transfer Protocol}
\newacronym{iot}{IoT}{Internet of Things}

\newacronym{https}{HTTPS}{HyperText Transfer Protocol Secure}
\newacronym{ssl}{SSL}{Secure Socket Layers}
\newacronym{xss}{XSS}{Cross Site Scripting}
\newacronym{ca}{CA}{Certificate Authority}
\newacronym{prng}{PRNG}{Pseudo Random Number Generator}
\newacronym{qca}{QSCA}{Quasi Static Code Analysis}
\newacronym{sca}{SCA}{Static Code Analysis}
\newacronym{sha}{SHA}{Secure Hash Algorithm}
\newacronym{dca}{DCA}{Dynamic Code Analysis}
\newacronym{pbe}{PBE}{Password Based Encryption}
\newacronym{tls}{TLS}{Transport Layer Security}
\newacronym{mitm}{MiTM}{Man in the Middle}
\newacronym{ml}{ML}{Machine Learning}
\newacronym{xml}{XML}{Extensible Markup Language}
\newacronym{yaml}{YAML}{YAML Ain't Markup Language}
\newacronym{csrf}{CSRF}{Cross-Site Request Forgery}
\newacronym{ssrf}{SSRF}{Server-Side Request Forgery}
\newacronym{dos}{DoS}{Denial of Service}
\newacronym{ddos}{DDoS}{Distributed Denial of Service}
\newacronym{jwt}{JWT}{Json Web Token}
\newacronym{ldap}{LDAP}{Lightweight Directory Access Protocol}
\newacronym{ecb}{ECB}{Electronic Codebook}
\newacronym{api}{API}{Application Program Interface}
\newacronym{cbc}{CBC}{Cipher Block Chaining}
\newacronym{mcc}{MCC}{Mccabe Cyclomatic Complexity}
\newacronym{aes}{AES}{Advanced Encryption Standard}
\newacronym{rsa}{RSA}{Rivest-Shamir-Adleman}
\newacronym{ast}{AST}{Abstract Syntax Tree}
\newacronym{iast}{IAST}{Interactive Application Security Testing}
\newacronym{soa}{SOA}{state-of-the-art}
\newacronym{uast}{UAST}{Universal Abstract Syntax Tree}
\newacronym{loc}{LoC}{Lines of Code}
\newacronym{sloc}{SLoC}{Source Lines of Code}
\newacronym{tp}{TP}{true positive}
\newacronym{fp}{FP}{false positive}
\newacronym{tn}{TN}{true negative}
\newacronym{fn}{FN}{false negative}
\newacronym{rce}{RCE}{Remote Code Execution}
\newacronym{ssh}{SSH}{Secure Shell Protocol}
\newacronym{sbom}{SBOM}{Software Bill Of Materials}
\newacronym{des}{DES}{Data Encryption Standard}
\newacronym{des3}{DES3}{Triple DES}
\newacronym{dsa}{DSA}{Digital Signature Algorithm}
\newacronym{csv}{CSV}{Comma Seperated Valutes}
\newacronym{json}{JSON}{JavaScript Object Notation}
\newacronym{url}{URL}{URL}
\newacronym{orbs}{ORBS}{Observation-Based Slicing}
\newacronym{mdw}{MDW}{Moving Deletion Window}
\newacronym{qses}{QSES}{Quasi-Static Executable Slices}
\newacronym{dll}{DLL}{Dynamically Linked Libraries}


\iffalse
USE THIS PACKAGE TO CHECK FOR UNUSED REFERENCES

\usepackage{refcheck}

ADD THIS LINE: \nocite{*}
\fi

\InputIfFileExists{Tables/tables}{
	%File is imported
}{
	%File isn't imported
}

%https://tex.stackexchange.com/questions/82531/how-to-change-hyperlinks-color-lyx
%Incase hyperlink colors aren't clear or black
\hypersetup{
	colorlinks=true,
	linkcolor=black,
	urlcolor=black,
	pdfauthor={ Miles Frantz },
	pdftitle={ Miles Frantz - Paper Submission },
	pdfsubject={ Miles Frantz - Paper Submission },
	pdfkeywords={Static Code Analysis, Code Analysis, Dynamic Code Analysis}
}
\def\BibTeX{{\rm B\kern-.05em{\sc i\kern-.025em b}\kern-.08em
T\kern-.1667em\lower.7ex\hbox{E}\kern-.125emX}}

\definecolor{codegreen}{rgb}{0,0.6,0}
\definecolor{codegray}{rgb}{0.5,0.5,0.5}
\definecolor{codepurple}{rgb}{0.58,0,0.82}
\definecolor{backcolour}{rgb}{0.95,0.95,0.92}
\definecolor{backstep}{HTML}{29c52d}
\definecolor{backsteptwo}{HTML}{0000ff}
\definecolor{backstepthree}{HTML}{ff5aff}
\definecolor{deepgreen}{rgb}{0,0.5,0}
\colorlet{stringcolour}{red!60!black}

%https://tex.stackexchange.com/questions/587558/box-around-the-code
%CODE LISTING :> \begin{lstlisting}[frame=single]
\lstdefinestyle{mystyle}{
        basicstyle=\ttfamily\footnotesize,
        keywordstyle=\color{backsteptwo}\ttfamily,
        otherkeywords={\%, \}, \{, \&, \|, self, True, False},
        sensitive=true,
        stringstyle=\color{stringcolour}\ttfamily,
        commentstyle=\color{backstep}\ttfamily,
        morecomment=[l][\color{backstep}]{\#},
        breakatwhitespace=false,
        emph={},
        emphstyle=\color{backstepthree},
        breaklines=true,
        captionpos=b,
        keepspaces=true,
        numbers=left,
        numbersep=2pt,
        showspaces=false,
        showstringspaces=false,
        showtabs=false,
        escapechar=@,
        columns=fullflexible,
        xleftmargin=0pt,
        tabsize=1,
        captionpos=b
}
\lstset{style=mystyle}

%\begin{python}
%\end

%https://tex.stackexchange.com/questions/157389/how-to-center-column-values-in-a-table
\newcolumntype{P}[1]{>{\centering\arraybackslash}p{#1}}
\newcolumntype{M}[1]{>{\centering\arraybackslash}m{#1}}

\def\anon{1}

\if\anon0
    \newcommand{\GitHub}{https://github.com/franceme/\MakeLowercase{\name}}
    \newcommand{\PyPi}{https://pypi.org/project/\MakeLowercase{\name}}
    \newcommand{\benchmarkGitHub}{https://github.com/franceme/\benchmark}
\else
    \newcommand{\GitHub}{https://github.com/\MakeLowercase{\name}/\MakeLowercase{\name}}
    \newcommand{\PyPi}{}
    \newcommand{\benchmarkGitHub}{https://github.com/\MakeLowercase{\name}/\benchmark}
\fi

\algnewcommand\LeftComment[2]{%
\hspace{#1\algindent}$\triangleright$ \eqparbox{COMMENT}{#2} \hfill %
}
%\algnewcommand{\LeftComment}[1]{\Statex \(\triangleright\) #1}
\newcommand{\hrulealg}[0]{\vspace{1mm} \hrule \vspace{1mm}}

\newcommand{\cut}[1]{\sout{\textcolor{red}{#1}}}
\newcommand{\espace}{\hspace{0.25em}}
\newcommand{\q}[1]{``#1''}
\newcommand{\cmt}[1]{}
\newcommand{\selfrep}[1]{\hyperref[#1]{#1}}
\newcommand{\qq}[1]{`#1'}
\newcommand{\acro}[1]{\textbf{\underline{#1}}}
\newcommand{\loref}[1]{\footnote{\href{#1}}}
\newcommand{\ts}{\textsuperscript}
\newcommand{\citeref}[2]{#1 \footnote{\href{#1}{#2}}}
\newcommand{\citeacro}[2]{#1 \ref{key:#1}}
\newcommand{\img}[1]{Imgs/#1}
\newcommand{\Code}[1]{code/#1}


\def\iz|#1#2|{\underline{#1}#2}
\newcommand{\lstBug}{\includegraphics[scale=.015]{\img{bug.png}}}
\newcommand{\lstCheckMark}{\includegraphics[scale=.015]{\img{checkmark.png}}}
\newcommand{\lstRightArrow}{\includegraphics[scale=.040]{\img{right_arrow.png}}}
\newcommand{\lstRedRightArrow}{\includegraphics[scale=.040]{\img{red_right_arrow.png}}}
\newcommand{\lstGreenCheckMark}{\includegraphics[scale=.040]{\img{green_checkmark.png}}}
\newcommand{\ghostSection}[1]{\smallskip\noindent\textbf{#1}}

\newcommand{\mybib}[1]{
	\bibliographystyle{IEEEtran}
	\bibliography{#1}
}

\newcommand{\mybibno}[1]{
	\nocite{*}
	\bibliographystyle{IEEEtran}
	\bibliography{#1}
}

%https://tex.stackexchange.com/questions/542536/how-can-i-output-only-the-first-letter-of-the-argument-of-a-command
\def\firstWord#1#2@{#1}
\def\restWord#1#2@{#2}
\newcommand{\showAbbreviation}[1]{\MakeUppercase{\underline{\firstWord#1@}}\MakeLowercase{\restWord#1@}}

\iffalse
it's is a contraction, meaning it is, it's going to rain
its is possessive, belonging to or a quality of it, its color is red
its' does not exist


A module is a set of functions, types, classes, ... put together in a common namespace.

A library is a set of modules which makes sense to be together and that can be used in a program or another library.

A package is a unit of distribution that can contain a library or an executable or both. It's a way to share your code with the community.



#######> Example latex usage

If based on anony or not
\name\if\anon0\footnote{Available from PyPi at \selfrep{\PyPi}}\fi\footnote{The source code available from GitHub at \selfrep{\GitHub}}

Source Code
\begin{figure}[H]
    \centering\hspace*{-0.75in}
    	\begin{lstlisting}[language=Python,escapechar=@,caption={The curry programming pattern for Python. The underlined code is inserted in a new copy of the AST.}, captionpos=b,label={lst:python_curry}]
    def top_call(num):
    	@\underline{num = 25} \label{curry_lsting:num_infer}@
    
    	def sub_call(sub_num): 
    		@\bug@ @\underline{sub\_num = 4} \label{curry_lsting:sub_num_infer}@
    		return sub_num * num
    
    	return sub_call
    
    overhead = top_call(25) @\label{curry_lsting:top_method_call}@
    overhead(4) @\label{curry_lsting:bottom_method_call}@
    \end{lstlisting}
\end{figure}

Referencing source code line		
\ref{curry_lsting:top_method_call}

Pulling Information from a raw source code
\begin{figure}[H]
    \centering\hspace*{-0.75in}
        \lstinputlisting[language=Python,escapechar=@,caption={The curry programming pattern for Python. The underlined code is inserted in a new copy of the AST.}, captionpos=b,label={lst:python_curry},firstline=XXX,lastline=XXX]{Code/Licma.py}
\end{figure}

A kind of section
\ghostSection{Builder}

Algorithms
%https://tex.stackexchange.com/questions/435617/glossaries-expand-acronyms-for-first-time-use-within-each-chapter

\begin{algorithm}
    \caption{The program flow of \name.} \label{algo:ProgramFlow}
    \begin{algorithmic}[1]
        \Procedure{MyProcedure}{}
        \State $\textit{stringlen} \gets \text{length of }\textit{string}$
        \State $i \gets \textit{patlen}$
        \BState \emph{top}:
        \If {$i > \textit{stringlen}$} \Return false
        \EndIf
        \State $j \gets \textit{patlen}$
        \BState \emph{loop}:
        \If {$\textit{string}(i) = \textit{path}(j)$}
        \State $j \gets j-1$.
        \State $i \gets i-1$.
        \State \textbf{goto} \emph{loop}.
        \State \textbf{close};
        \EndIf
        \State $i \gets i+\max(\textit{delta}_1(\textit{string}(i)),\textit{delta}_2(j))$.
        \State \textbf{goto} \emph{top}.
        \EndProcedure
    \end{algorithmic}
\end{algorithm}
\fi
